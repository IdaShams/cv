%%%%%%%%%%%%%%%%%%%%%%%%%%%%%%%%%%%%%%%%%
% Twenty Seconds Resume/CV
% LaTeX Template
% Version 1.0 (14/7/16)
%
% Original author:
% Carmine Spagnuolo (cspagnuolo@unisa.it) with major modifications by 
% Vel (vel@LaTeXTemplates.com) and Harsh (harsh.gadgil@gmail.com)
%
% License:
% The MIT License (see included LICENSE file)
%
%%%%%%%%%%%%%%%%%%%%%%%%%%%%%%%%%%%%%%%%%

\documentclass[a4paper]{twentysecondcv}
\cvname{Paul Freeman}
\cvjobtitle{Computer Scientist}
\cvlinkedin{linkedin.com/in/freemapa}
\cvgithub{github.com/paul-freeman}
\cvnumberphone{+64 0210416766}
\cvsite{anaconda.org/freemapa}{pypi.org/user/freemapa}
\cvmail{paul.freeman.cs@gmail.com}

\programming{{JavaScript $\textbullet$ Lua $\textbullet$ Go / 3},
             {C++ $\textbullet$ Java $\textbullet$ Kotlin / 3.5},
             {Python $\textbullet$ C $\textbullet$ Haskell $\textbullet$ Elm / 5}}

\newcommand\skills{ 
    \smartdiagram[bubble diagram]{
        \textbf{Scientific}\\\textbf{Computing},
        \textbf{Full Stack}\\\textbf{Web Dev},
        \textbf{Functional}\\\textbf{Programming},
        \textbf{Machine}\\\textbf{Learning},
        \textbf{Automation},
        \textbf{Statistical}\\\textbf{Analysis}
    }
}

\projects{\textbf{PLACE}
          A modular laboratory automation web app, using a combination of
          Python, Elm, and JavaScript

          \textbf{DTW}
          A fast Cython implementation of the Dynamic Time Warping algorithm,
          designed to replace a slower R dependency

          \textbf{RUS}
          Python and C implementation of forward and inverse algorithms for
          Resonant Ultrasound Spectroscopy

          \textbf{MUN33BOT}
          Autonomous financial management system which uses spending patterns
          and statistical analysis to learn and provide future spending
          recommendations
}

\begin{document}
\makeprofile{}

\section{Experience}
\begin{twenty}
\twentyitem{2016--2018}
           {}
           {Software Engineer}
           {\href{https://pal.auckland.ac.nz/}{UoA - Physical Acoustics Laboratory}}
           {}
           {Full stack development of a modular laboratory automation
           framework: from design; to implementation; to rollout, training, and
           maintenance.
           Skills: \emph{Python, Elm, Django, JavaScript, Github, C, Cython, JSON,
           Java, Lua, websockets, CSS, conda, PyPi, NumPy}} \\

\twentyitem{2014--2016}
           {}
           {Computer Science Tutor}
           {\href{http://www.auckland.ac.nz/}{UoA - Computer Science Dept}}
           {}
           {Educated computer science students as the solo tutor for the
           Assembly and C programming course. Redeveloped the course slides and
           tutor materials to increase student engagement.
           Skills: \emph{Assembly, C, Linux, Vim}} \\

\twentyitem{2013--2013}
           {}
           {Software Engineer}
           {\href{https://www.intel.com}{Intel - High Performance Computing}}
           {}
           {Developed low-level software development tools to perform debug
           tasks on many-core CPUs.
           Skills: \emph{C, Assembly, Github, Linux, Bash}} \\

\twentyitem{2012--2012}
           {}
           {Software Engineer}
           {\href{http://www.garmin.com}{Garmin AT - Navigation/Communication Team}}
           {}
           {Implemented an FAA-compliant software testing framework used to
           debug embedded code within aeronautical navigation hardware.
           Skills: \emph{\CC, StarTeam, Jira, Lua,~.NET, Perl}} \\

\twentyitem{2011--2014}
           {}
           {Computer Science Teaching Assistant}
           {\href{http://www.oregonstate.edu/}{OSU - Computer Science Dept}}
           {}
           {Assisted with a wide range of courses, including: microcontroller
           programming, C development, Linux systems programming, translators
           \& compilers, and Qt development.
           Skills: \emph{Assembly, C, Linux, Qt, Haskell, C\#}} \\

\twentyitem{2010--2011}
           {}
           {Software Validation Intern}
           {\href{https://www.becpdx.org/}{Business Education Compact - Intel}}
           {}
           {Developed and executed validation tests for embedded software on 2D
           and 3D flat panel televisions.
           Skills: \emph{C, Bash, Linux, Subversion}} \\

\twentyitem{1997--2010}
           {}
           {Retail Manager}
           {\href{https://www.target.com/}{Target},
           \href{https://en.wikipedia.org/wiki/CompUSA}{CompUSA},
           \href{https://en.wikipedia.org/wiki/Hastings\_Entertainment}{Hastings},
           \href{https://www.staples.com/}{Staples}}
           {}
           {Prior to studying computer science and graphic design, worked
           successfully within the retail sector. With a variety of employers,
           consistently excelled in leadership roles, managing teams of 5--50
           employees.
           Skills: \emph{management, leadership, communication, sales}}

\end{twenty}

\section{Education}
\begin{twenty}
    \twentyitem{2014--2016}{}{M.S., Computer Science}{\href{https://www.cs.auckland.ac.nz/en.html}{University of Auckland}}{}{First Class Honours}
    \twentyitem{2010--2014}{}{B.S., Computer Science}{\href{http://eecs.oregonstate.edu/academics/undergraduates/computer-science}{Oregon State University}}{}{Summa Cum Laude}
    \twentyitem{2008--2010}{}{A.A., Graphic Design}{\href{https://www.pcc.edu/programs/graphic-design/}{Portland Community College}}{}{Highest Honors}
\end{twenty}

\section{Publications}
Freeman, P. F. (2016). \emph{Abstract Syntax Tree Retrieval: Inferring Student
Coding Goals Using Case-based Reasoning and Code Similarity}. The University of
Auckland. ResearchSpace@Auckland.

Freeman, P., Watson, I., \& Denny, P. (2016). \emph{Inferring Student Coding
Goals Using Abstract Syntax Trees}. Paper presented at 24th International
Conference on Case-Based Reasoning Research and Development (ICCBR). Atlanta, GA.\@

Han, K., Freeman, P., Han, H.-Y., Hamar, J., Stack Jr., J. F. (2014).
\emph{Finite-Difference Time-Domain Modeling of Ultra-High Frequency Antennas
on and Inside the Carbon Fiber Body of a Solar-Powered Electric Vehicle}. ACES
Journal, Vol. 29, No. 6.

\end{document} 
